\chapter{Goals}
\label{chap:goals}

% soll hier noch gesagt werden, dass wir davon ausgehen, dass es keine endzust�nde gibt?

Based on the purpose to bring the healthcare applications of the University of Auckland's healthcare project and certain safety standards closer together, and thereby to lay the foundation of aspired safety certifications, Robostudio is to be extended with therefor essential tools. The visual programming environment is used for developing robotic applications in the form of state machines and shall provide a new interface in the future, which can be used by the developer for evaluating safety functionality of robot behaviour. On the one hand this extension shall be seamlessly integrated in Robostudio. On the other hand it should provide a clear API which allows easy integration also into other program environments. Thus the new concepts are to be universal and not confined to Robostudio.

We want to introduce a visual language for defining safety constraints in state machine definitions of robot behaviour, that is suitable for people who are not experts in formal methods. The visual language should be easy to use and intuitive to read, in order to facilitate maintainability.
Hence, the graphical formalism should abstract from the usually complex and mathematical concept of conventional temporal logic but still be significantly expressive.
 
Furthermore we want to integrate some mechanism to support users in finding suitable constraints for a certain application. We assume that applications are modeled by state transition graphs. This modeling paradigm is used in many healthcare robots that employ dialogue systems for communicating with users~\cite{Bickmore2006}. Other possible fields might be shop assistant robots or entertainment robots. For such systems, our approach is suitable. It is based on heuristically analyzing state transition graphs and deriving constraints related to the graph structure. In this work, we want to investigate a particular heuristic as a first step towards user support. 

Finally, we will present some results of applying this heuristics to the medication reminder example described before. The minimum goal that should be achieved was the identification of the following constraints:

\begin{itemize}
	\item The application will always eventually check the database for new reminder jobs. Thus we ensure that pending reminders are processed.
%	\item Either a reminded medication is taken properly, or caregivers are notified that the patient refused medication intake.
	\item Whenever there is a pending reminder, medication will be finally taken or caregivers will get notified in case of the patient refusing medication intake.
% The next item is left out so that we can say, our heuristic found one we didn't think about yet :)
%	\item During the medication intake process the program can not switch to other services such as entertainment mode ore blood pressure measurement, for example.
\end{itemize}

Some of the statements do not make sense if patients lie or pretend to take their medication but don't take it. As our approach validates state machine behaviour rather than human attitudes we assume there is either cooperation from the patient or a separate tool that can verify medication intake. Furthermore we assume the correctness of external services used by the medication reminder such as the database module holding all reminders.

All the functionality mentioned above shall be combined in one editor and for evaluation purpose later be integrated into Robostudio, a programming environment for state machine editing.