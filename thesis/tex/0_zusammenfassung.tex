\chapter*{Zusammenfassung}

Robotik-Anwendungen sind heutzutage nicht mehr nur in der Industrie zu finden, sondern zunehmend auch im menschlichen Bereich und Lebensumfeld, wo die Roboter in direkter N�he zu Menschen arbeiten. Auch die Programmierung solcher Roboter wird nicht mehr zwangsl�ufig von Softwareentwicklern erledigt. Der Kunde m�chte das Verhalten der Roboter je nach deren Einsatzgebiet individuell selbst definieren und anpassen k�nnen. Doch solche Roboter im Umfeld von Menschen stellen aufgrund der Verletzungsgefahr ein Sicherheitsrisiko dar, zumal wenn das Roboterverhalten nicht von geschulten Ingenieuren entwickelt wurde. Daher ist nach L�sungen zu suchen, wie man mit Sicherheitsanforderungen f�r Roboter im Zusammenhang mit kundeneigener Programmierung umgeghen soll.

Mit dieser Arbeit wird eine visuelle Sprache vorgestellt, die der Erstellung von Sicherheitseigenschaften f�r zustandsbasierte Roboterprogramme dient.
Bei der Entwicklung dieser Sprache wurde besonders darauf Wert gelegt, dass auch Laien, die insbesondere nicht mit Sicherheitsaspekten vertraut sind, damit umgehen und arbeiten k�nnen. Dazu wurde zu einer mathematischen temporalen Beschreibungssprache eine intuitive visuelle Darstellung erarbeitet, die einem gr��eren Personenkreis Zugang zu Methoden der Programmverifikation verschaffen soll.
Au�erdem wird in dieser Arbeit ein neues Konzept vorgestellt, das mittels einer automatischen Bestimmung von Sicherheitsbedingungen den Entwickler bei der Programmierung unterst�tzt. Es soll dem Entwickler dabei helfen wichtige Bedingungen zu finden, und letztendlich die Wahrscheinlichkeit erh�hen, dass Programmfehler gefunden werden.

Ein graphischer Editor wurde entwickelt, der die visuelle Sprache zur Definition und Bearbeitung von Sicherheitsbedingungen wie auch die automatisierte Bedingungsfindung integrierend zusammenf�hrt und gemeinsam nutzbar macht.